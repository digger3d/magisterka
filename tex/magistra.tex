\documentclass{article}

\usepackage[utf8]{inputenc}
\usepackage{polski}
\usepackage{natbib}

\begin{document}

\title{Moja wspaniała magisterka}
\author{Konrad Solarz\\Uniwerstytet Warszawski}
\date{Czerwiec 2015}
\maketitle
\clearpage
\begin{abstract}
asf

\end{abstract}
\clearpage

\tableofcontents
\clearpage

\section{Motywacja pracy}
Badania nad kolcami dendrytycznymi sa bardzo ważne ze względu na ich postulowaną rolę w procesie uczenia się % cos z bazy ile badan itp.
Problem nabywanie nowych umiejętności i adaptowania się do nowych warunków jest jednym ze starszych zagadnień w naukach o mózgu.
Wielu z naukowców uważa, że klucz do zrozumienia wspomnianego procesu leży właśnie w mechanizmach sterujących wzrostem i zanikaniem kolców dendrytycznych.
Niestety obecne metody analizowania danych eksperymentalnych są bardzo czasochłonne i wymagają zaangażowania wyspecjalizowanego personelu, przez co współczesne rozwiązania są słabo skalowalne.
W dobie szybkich i tanich komputerów, coraz bardziej zaawansowanych metod uczenia maszynowego, przetwarzania obrazów, przechowywania i udostępniania wyników, naturalnym jest zwrócenie się ku automatyzacji tychże procesów.
Gdyby udało się zastąpić naukowca przez komputer przy żmudnym i długotrwałym procesie analizy danych pozwoliłoby to na dużą oszczędność czasu i zasobów, przyspieszyłoby pierwsze etapy analizy wyników.
Zważając na powyższe, mogłoby to skutkować przyspieszeniem procesu badawczego i/lub zwiększeniem liczebności próby poddanej eksperymentowi i analizie.
Celem niniejszej pracy magisterskiej jest wypracowanie metodyki pozwalającej na zautomatyzowanie, przyspieszenie i zobiektywizowanie procesu analizy zdjęć kolców dendrytycznych uzyskanych metodami mikroskopii konfokalnej.



\section{Wstęp}
Konlce dendrytyczne zostały po raz pierwszy opisane przez Ramona y Cajala ponad 100 lat temu. %% źródło
Skupiły na sobie zainteresowanie od samego początku, jednak rozwój technik badawczych (mikroskopia konfokalna, dwufotoonowa,
glutamate caging technique) % - technika aktywowania światłem?)
pozwolił na zintensyfikowanie badań nad kolcami dendrytycznymi. W następnych podrozdziałach opisana zostanie postulowana rola
biologiczna kolców dendrytycznych, a także pokrótce technika mikroskopowa, która posłużyła do wykonania fotografii stanowiących źródło danych dla niniejszej pracy. 
    
\subsection{Biologiczna rola kolców dendrytycznych}
Większość zakończeń postsynaptycznych %% kulawe tlumaczenie?
w ludzkim mózgu znajduje się na czubkach kolców dendrytycznych.
Naukowcy wskazują kilka udogodnień, które oferuje takie rozwiązanie.
Zostaną one opisane w dalszych rozdziałach pracy.
\subsubsection{Przedział biochemiczny}
Komórka jest wielce skomplikowaną maszyną biochemiczną. 
Wiele z jej mechamnizmów regulacyjnych opiera się na róźnicy stężeń w jej częściach. 
Jak wykazały badania %cytat
jednym z kluczowych zadań kolców dendrytycznych jest utrzymywanie odmiennego środowiska biochemicznego od reszty neuronu.
Uczenie opiera się w dużym stopniu na kaskadzie zmian w ekspresji genów i zachowania białek wywołanej przez napływ jonów wapnia.
Naukowcy zaobserwowali, że podczas pobudzenia postsynaptycznego obserwuje się wysoki wzrost koncentracji jonów Ca<sup>{2+}</sup> w obrębie samego kolca, znikomy zaś w połączonym z nim dendrytem.
Zaangażowany może być w to proces utrudniania dyfuzji białek kluczowych dla regulacji stężenia jonów wapnia.
W porównaniu do cząsteczek o podobnych gabarytach w obrębie kolca dendrytycznego lepkość cytozolu wydaje się być selektywnie wyższa dla wybranych białek.
Większość cząsteczek przytwierdzonych do błony komórkowej dość swobodnie się przemieszcze w jej obrębie. 
Zachowanie takie byłoby wysoce niepożądane w przypdaku zakończeń synaptycznych. 
Swobodnie dyfundujące receptory cząsteczek sygnałowych wysyłanych przez akson neuronu aferentnego obniżałyby efektywność przesyłu informacji przez taką synapsę. 
Kolce dendrytyczne utrudniają swobodne przemieszczanie się receptorów neurotransmiterów, przez co synapsy znajdujące się w obrębie tych struktur mają do dyspozycji większą pulę białek receptorowych, co prawdopodobnie wpływa na ich siłę i stabilność. 

\subsubsection{Przedział elektryczny}
Jednych z kanałów przekazu informacji w obrębie komórki nerwowej jest impuls elektryczny.
Podobnie jak przypadku składu biochemicznego cytozolu i tutaj kolce potrafią utrzymać odmienne napięcie elektryczne na błonie samego kolca od tego panującego w reszcie dendrytu.
Początkowo wydawało się, że szyja kolca dendrytycznego oferuje zbyt mały opór, by w znaczący sposób odizolować kolec dendrytyczny od komórki. 
Jednak ostatnie badania wykazały, że przynajmniej niektóre z kolców dendrytycznych podtrafią odizolować głowę kolca od dendrytu.
Predystynowane do tego wydają się być kolce o szyjach dlugich i cienkich.

\subsubsection{Plastyczność kolców dendrytycznych}
Skoro w kolcach dendrytycznych upatruje się pierwszoplanowych aktorów w procesie uczenia, to powinno się dać zaobserwowac zmiany zachodzące podczas treningu.
Istotnie, wiele eksperymentów potwierdza taką hipotezę.
Długotrwałym wzmocnieniem synapsy (\emph{LTP - long term potentiation}) nazywa sie taką sytuację, że taka sama stymulacja komórki presynaptycznej powoduje większe pobudzenie komórki postsynaptycznej.
Opisane zjawisko traktuje się jako komórkowy substrat uczenia się.
Badacze indukujący w neuronach LTP zaobserwowali zwiększanie się objętości kolców dendrytycznych, w obrębie których leżały pobudzane synapsy.
Co więcej, kolce większe były zarówno strukturalnie jak i funkcjonalnie stabilniejsze.
Co ciekawe, jony wapniowe, które są niezbędne dla modyfikacji siły synaps, spełniają także ważną rolę w regulacji procesu wzrostu kolców dendrytycznych.
Ułatwiają one polimeryzację aktyny, co prowadzi do wzrostu cytoszkieletu, w wyniku czego zwiększa się objętość kolca. 

\subsubsection{Morfologia Kolców Dendrytycznych}
Neurobiolodzy wyróźniają trzy główne grupy kolców dedrytycznych: kolce grzybkowate, cienkie i długie, oraz przysadziste (\emph{ang. stubby})
W pierwszej grupie kolców można rozróźnić dwie dobrze wyodrębnione struktury: główkę oraz szyjkę.
Głowa ma zaokrąglony kształt i na jejczubku znajduje się zgrubienie postsynaptyczne.
W błonie znajdują się receptory (przede wszystkim AMPA oraz NMDA) zaś w płynie komórkowym białka odpowiedzialne za regulację sprzężenia kolca dendrytycznego z synapsą oraz resztą dendrytu. %% czy aby tutaj nie pieprzę?
W kolcu długim i cienkim typowo nie można wyróźnić główki, a kształt przypomina długie filopodium.
W typie przysadzistym także, trudno jest zlokalizować okrągłe zwieńczenie szyi, gdyż w kolce należące do tej grupy przypominają krótkie, acz grube wypustki dendrytu. %% dodać koniecznie grafikę 
\cite{Sala2014}
\cite{Costa2007}

\subsection{Mikroskopia konfokalna}
vddd

\section{Dane}
\section{Metody}
\subsection{Klastrowanie hierarchiczne}
\subsection{Algorytm K-means}
\subsection{Indeksy}
\section{Wyniki}
\section{Dyskusja}

\bibliographystyle{plain}

\bibliography{magisterka}

\end{document}

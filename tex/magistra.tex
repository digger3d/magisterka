\documentclass{article}

\usepackage[utf8]{inputenc}
\usepackage{polski}


\begin{document}

\title{Moja wspaniała magisterka}
\author{Konrad Solarz\\Uniwerstytet Warszawski}
\date{Czerwiec 2015}
\maketitle
\clearpage
\begin{abstract}
asf

\end{abstract}
\clearpage

\tableofcontents
\clearpage

\section{Motywacja pracy}
Badania nad kolcami dendrytycznymi sa bardzo ważne ze względu na ich postulowaną rolę w procesie uczenia się % cos z bazy ile badan itp.
Problem nabywanie nowych umiejętności i adaptowania się do nowych warunków jest jednym ze starszych zagadnień w naukach o mózgu.
Wielu z naukowców uważa, że klucz do zrozumienia wspomnianego procesu leży właśnie w mechanizmach sterujących wzrostem i zanikaniem kolców dendrytycznych.
Niestety obecne metody analizowania danych eksperymentalnych są bardzo czasochłonne i wymagają zaangażowania wyspecjalizowanego personelu, przez co współczesne rozwiązania są słabo skalowalne.
W dobie szybkich i tanich komputerów, coraz bardziej zaawansowanych metod uczenia maszynowego, przetwarzania obrazów, przechowywania i udostępniania wyników, naturalnym jest zwrócenie się ku automatyzacji tychże procesów.
Gdyby udało się zastąpić naukowca przez komputer przy żmudnym i długotrwałym procesie analizy danych pozwoliłoby to na dużą oszczędność czasu i zasobów, przyspieszyłoby pierwsze etapy analizy wyników.
Zważając na powyższe, mogłoby to skutkować przyspieszeniem procesu badawczego i/lub zwiększeniem liczebności próby poddanej eksperymentowi i analizie.
Celem niniejszej pracy magisterskiej jest wypracowanie metodyki pozwalającej na zautomatyzowanie, przyspieszenie i zobiektywizowanie procesu analizy zdjęć kolców dendrytycznych uzyskanych metodami mikroskopii konfokalnej.



\section{Wstęp}

\subsection{Biologiczna rola kolców dendrytycznych}
sddd

\subsection{Mikroskopia konfokalna}
vddd
\end{document}

\documentclass{article}

\usepackage[utf8]{inputenc}
\usepackage{polski}


\begin{document}

\title{Moja wspaniała magisterka}
\author{Konrad Solarz\\Uniwerstytet Warszawski}
\date{Czerwiec 2015}
\maketitle
\clearpage
\begin{abstract}
asf

\end{abstract}
\clearpage

\tableofcontents
\clearpage

\section{Motywacja pracy}
Badania nad kolcami dendrytycznymi sa bardzo ważne ze względu na ich postulowaną rolę w procesie uczenia się % cos z bazy ile badan itp.
Problem nabywanie nowych umiejętności i adaptowania się do nowych warunków jest jednym ze starszych zagadnień w naukach o mózgu.
Wielu z naukowców uważa, że klucz do zrozumienia wspomnianego procesu leży właśnie w mechanizmach sterujących wzrostem i zanikaniem kolców dendrytycznych.
Niestety obecne metody analizowania danych eksperymentalnych są bardzo czasochłonne i wymagają zaangażowania wyspecjalizowanego personelu, przez co współczesne rozwiązania są słabo skalowalne.
W dobie szybkich i tanich komputerów, coraz bardziej zaawansowanych metod uczenia maszynowego, przetwarzania obrazów, przechowywania i udostępniania wyników, naturalnym jest zwrócenie się ku automatyzacji tychże procesów.
Gdyby udało się zastąpić naukowca przez komputer przy żmudnym i długotrwałym procesie analizy danych pozwoliłoby to na dużą oszczędność czasu i zasobów, przyspieszyłoby pierwsze etapy analizy wyników.
Zważając na powyższe, mogłoby to skutkować przyspieszeniem procesu badawczego i/lub zwiększeniem liczebności próby poddanej eksperymentowi i analizie.
Celem niniejszej pracy magisterskiej jest wypracowanie metodyki pozwalającej na zautomatyzowanie, przyspieszenie i zobiektywizowanie procesu analizy zdjęć kolców dendrytycznych uzyskanych metodami mikroskopii konfokalnej.



\section{Wstęp}
Konlce dendrytyczne zostały po raz pierwszy opisane przez Ramona y Cajala ponad 100 lat temu. %% źródło
Skupiły na sobie zainteresowanie od samego początku, jednak rozwój technik badawczych (mikroskopia konfokalna, dwufotoonowa,
glutamate caging technique) % - technika aktywowania światłem?)
pozwolił na zintensyfikowanie badań nad kolcami dendrytycznymi. W następnych podrozdziałach opisana zostanie postulowana rola
biologiczna kolców dendrytycznych, a także pokrótce technika mikroskopowa, która posłużyła do wykonania fotografii stanowiących źródło danych dla niniejszej pracy. 
    
\subsection{Biologiczna rola kolców dendrytycznych}
Większość zakończeń postsynaptycznych %% kulawe tlumaczenie?
w ludzkim mózgu znajduje się na czubkach kolców dendrytycznych.
Naukowcy wskazują kilka udogodnień, które oferuje takie rozwiązanie.
Zostaną one opisane w dalszej części pracy.
\subsubsection{Przedział biochemiczny}
Komórka jest wielce skomplikowaną maszyną biochemiczną. 
Wiele z jej mechamnizmów regulacyjnych opiera się na róźnicy stężeń w jej częściach. 
Jak wykazały badania %cytat
jednym z kluczowych zadań kolców dendrytycznych jest utrzymywanie odmiennego środowiska biochemicznego od reszty neuronu.
Uczenie opiera się w dużym stopniu na kaskadzie zmian w ekspresji genów i zachowania białek wywołanej przez napływ jonów wapnia.
Naukowcy zaobserwowali, że podczas pobudzenia postsynaptycznego obserwuje się wysoki wzrost koncentracji jonów Ca2+ %% wyedytować
w obrębie samego kolca, znikomy zaś w połączonym z nim dednrytem. 
\subsubsection{Morfologia Kolców Dendrytycznych}
Neurobiolodzy wyróźniają trzy główne grupy kolców dedrytycznych: kolce grzybkowate, cienkie i długie, oraz przysadziste (\emph{ang. stubby}).
W pierwszej grupie kolców można rozróźnić dwie dobrze wyodrębnione struktury: główkę oraz szyjkę.
Głowa ma zaokrąglony kształt i na jejczubku znajduje się zgrubienie postsynaptyczne.
W błonie znajdują się receptory (przede wszystkim AMPA oraz NMDA) zaś w płynie komórkowym białka odpowiedzialne za regulację sprzężenia kolca dendrytycznego z synapsą oraz resztą dendrytu. %% czy aby tutaj nie pieprzę?
W kolcu długim i cienkim typowo nie można wyróźnić główki, a kształt przypomina długie filopodium.
W typie przysadzistym także, trudno jest zlokalizować okrągłe zwieńczenie szyi, gdyż w kolce należące do tej grupy przypominają krótkie, acz grube wypustki dendrytu. %% dodać koniecznie grafikę 


\subsection{Mikroskopia konfokalna}
vddd

\section{Dane}
\section{Metody}
\subsection{Klastrowanie hierarchiczne}
\subsection{Algorytm K-means}
\subsection{Indeksy}
\section{Wyniki}
\section{Dyskusja}]
\end{document}

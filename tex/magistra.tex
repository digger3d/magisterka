\documentclass{article}

\usepackage[utf8]{inputenc}
\usepackage{polski}
\usepackage{natbib}
\usepackage{ amssymb }
\begin{document}

\title{Moja wspaniała magisterka}
\author{Konrad Solarz\\Uniwerstytet Warszawski}
\date{Czerwiec 2015}
\maketitle
\clearpage
\begin{abstract}
asf

\end{abstract}
\clearpage

\tableofcontents
\clearpage

\section{Motywacja pracy}
Badania nad kolcami dendrytycznymi sa bardzo ważne ze względu na ich postulowaną rolę w procesie uczenia się % cos z bazy ile badan itp.
Problem nabywanie nowych umiejętności i adaptowania się do nowych warunków jest jednym ze starszych zagadnień w naukach o mózgu.
Wielu z naukowców uważa, że klucz do zrozumienia wspomnianego procesu leży właśnie w mechanizmach sterujących wzrostem i zanikaniem kolców dendrytycznych.
Niestety obecne metody analizowania danych eksperymentalnych są bardzo czasochłonne i wymagają zaangażowania wyspecjalizowanego personelu, przez co współczesne rozwiązania są słabo skalowalne.
W dobie szybkich i tanich komputerów, coraz bardziej zaawansowanych metod uczenia maszynowego, przetwarzania obrazów, przechowywania i udostępniania wyników, naturalnym jest zwrócenie się ku automatyzacji tychże procesów.
Gdyby udało się zastąpić naukowca przez komputer przy żmudnym i długotrwałym procesie analizy danych pozwoliłoby to na dużą oszczędność czasu i zasobów, przyspieszyłoby pierwsze etapy analizy wyników.
Zważając na powyższe, mogłoby to skutkować przyspieszeniem procesu badawczego i/lub zwiększeniem liczebności próby poddanej eksperymentowi i analizie.
Celem niniejszej pracy magisterskiej jest wypracowanie metodyki pozwalającej na zautomatyzowanie, przyspieszenie i zobiektywizowanie procesu analizy zdjęć kolców dendrytycznych uzyskanych metodami mikroskopii konfokalnej.



\section{Wstęp}
Konlce dendrytyczne zostały po raz pierwszy opisane przez Ramona y Cajala ponad 100 lat temu. %% źródło
Skupiły na sobie zainteresowanie od samego początku, jednak rozwój technik badawczych (mikroskopia konfokalna, dwufotoonowa,
glutamate caging technique) % - technika aktywowania światłem?)
pozwolił na zintensyfikowanie badań nad kolcami dendrytycznymi. W następnych podrozdziałach opisana zostanie postulowana rola
biologiczna kolców dendrytycznych, a także pokrótce technika mikroskopowa, która posłużyła do wykonania fotografii stanowiących źródło danych dla niniejszej pracy. 
    
\subsection{Biologiczna rola kolców dendrytycznych}
Większość zakończeń postsynaptycznych %% kulawe tlumaczenie?
w ludzkim mózgu znajduje się na czubkach kolców dendrytycznych.
Naukowcy wskazują kilka udogodnień, które oferuje takie rozwiązanie.
Zostaną one opisane w dalszych rozdziałach pracy.
\subsubsection{Przedział biochemiczny}
Komórka jest wielce skomplikowaną maszyną biochemiczną. 
Wiele z jej mechamnizmów regulacyjnych opiera się na róźnicy stężeń w jej częściach. 
Jak wykazały badania %cytat
jednym z kluczowych zadań kolców dendrytycznych jest utrzymywanie odmiennego środowiska biochemicznego od reszty neuronu.
Uczenie opiera się w dużym stopniu na kaskadzie zmian w ekspresji genów i zachowania białek wywołanej przez napływ jonów wapnia.
Naukowcy zaobserwowali, że podczas pobudzenia postsynaptycznego obserwuje się wysoki wzrost koncentracji jonów Ca<sup>{2+}</sup> w obrębie samego kolca, znikomy zaś w połączonym z nim dendrytem.
Zaangażowany może być w to proces utrudniania dyfuzji białek kluczowych dla regulacji stężenia jonów wapnia.
W porównaniu do cząsteczek o podobnych gabarytach w obrębie kolca dendrytycznego lepkość cytozolu wydaje się być selektywnie wyższa dla wybranych białek.
Większość cząsteczek przytwierdzonych do błony komórkowej dość swobodnie się przemieszcze w jej obrębie. 
Zachowanie takie byłoby wysoce niepożądane w przypdaku zakończeń synaptycznych. 
Swobodnie dyfundujące receptory cząsteczek sygnałowych wysyłanych przez akson neuronu aferentnego obniżałyby efektywność przesyłu informacji przez taką synapsę. 
Kolce dendrytyczne utrudniają swobodne przemieszczanie się receptorów neurotransmiterów, przez co synapsy znajdujące się w obrębie tych struktur mają do dyspozycji większą pulę białek receptorowych, co prawdopodobnie wpływa na ich siłę i stabilność. 

\subsubsection{Przedział elektryczny}
Jednych z kanałów przekazu informacji w obrębie komórki nerwowej jest impuls elektryczny.
Podobnie jak przypadku składu biochemicznego cytozolu i tutaj kolce potrafią utrzymać odmienne napięcie elektryczne na błonie samego kolca od tego panującego w reszcie dendrytu.
Początkowo wydawało się, że szyja kolca dendrytycznego oferuje zbyt mały opór, by w znaczący sposób odizolować kolec dendrytyczny od komórki. 
Jednak ostatnie badania wykazały, że przynajmniej niektóre z kolców dendrytycznych podtrafią odizolować głowę kolca od dendrytu.
Predystynowane do tego wydają się być kolce o szyjach dlugich i cienkich.

\subsubsection{Plastyczność kolców dendrytycznych}
Skoro w kolcach dendrytycznych upatruje się pierwszoplanowych aktorów w procesie uczenia, to powinno się dać zaobserwowac zmiany zachodzące podczas treningu.
Istotnie, wiele eksperymentów potwierdza taką hipotezę.
Długotrwałym wzmocnieniem synapsy (\emph{LTP - long term potentiation}) nazywa sie taką sytuację, że taka sama stymulacja komórki presynaptycznej powoduje większe pobudzenie komórki postsynaptycznej.
Opisane zjawisko traktuje się jako komórkowy substrat uczenia się.
Badacze indukujący w neuronach LTP zaobserwowali zwiększanie się objętości kolców dendrytycznych, w obrębie których leżały pobudzane synapsy.
Co więcej, kolce większe były zarówno strukturalnie jak i funkcjonalnie stabilniejsze.
Co ciekawe, jony wapniowe, które są niezbędne dla modyfikacji siły synaps, spełniają także ważną rolę w regulacji procesu wzrostu kolców dendrytycznych.
Ułatwiają one polimeryzację aktyny, co prowadzi do wzrostu cytoszkieletu, w wyniku czego zwiększa się objętość kolca. 

\subsubsection{Morfologia Kolców Dendrytycznych}
Neurobiolodzy wyróźniają trzy główne grupy kolców dedrytycznych: kolce grzybkowate, cienkie i długie, oraz przysadziste (\emph{ang. stubby})
W pierwszej grupie kolców można rozróźnić dwie dobrze wyodrębnione struktury: główkę oraz szyjkę.
Głowa ma zaokrąglony kształt i na jejczubku znajduje się zgrubienie postsynaptyczne.
W błonie znajdują się receptory (przede wszystkim AMPA oraz NMDA) zaś w płynie komórkowym białka odpowiedzialne za regulację sprzężenia kolca dendrytycznego z synapsą oraz resztą dendrytu. %% czy aby tutaj nie pieprzę?
W kolcu długim i cienkim typowo nie można wyróźnić główki, a kształt przypomina długie filopodium.
W typie przysadzistym także, trudno jest zlokalizować okrągłe zwieńczenie szyi, gdyż w kolce należące do tej grupy przypominają krótkie, acz grube wypustki dendrytu. %% dodać koniecznie grafikę 
\cite{Sala2014}
\cite{Costa2007}

\subsection{Mikroskopia konfokalna}
vddd

\section{Dane}
\section{Metody}
W świecie nauki i technologii w ostatnich latach pojawia się problem natłoku danych.
Potrafimy zbierać tak ogromne ilości danych, że człowiek nie jest w stanie ich przetworzyć.
Odpowiedzią na ten problem jest dzedzina zwana uczeniem maszynowym. 
Zajmuje się ona opracowywaniem algorytmów, które wyposażony w zbiór uczący, będą w stanie dostarczyć wyektrahowanej zeń informacji.
Obecnie najdroższym i najbardziej czasochłonnym etapem w procesie przetwarzania informacji jest praca człowieka.
O ile na ostatnich stopniach obróbki danych jego udział jest niezbędny czy wręcz pożądany, o tyle uciekanie się do pracy ludzkiej na pierwszych etapach procesowania danych znakomicie wydłużyłoby takie przedsięwzięcie, zwiększyło koszty, lub nawet z uwagi na dwa poprzednie czynniki zupełnie go uniemożliwiło.
Dlatego wiele wisiłku wkładane jest w takie opracowywanie algorytmów, aby nakład luzkiej pracy niezbędny do ich działania był minimalny.
W niniejszej pracy zastosowano szereg rozwiązań stosowanych na polu uczenia maszynowego.
Zostaną one opisane w poniższych rozdziałach.
\subsection{Klastrowanie hierarchiczne}

\subsection{Algorytm K-means}
\subsection{Uczenie pół-nadzorowane}
Dzisiaj zbieranie danych jest stosunkowo tanie.
Natomiast, żeby taki zbiór danych dało się dostarczyć na wejście jakiego algorytmu uczącego się, należy go uprzednio opisać.
Jeśli interesuje nas zagadnienie klasyfikacji, należy obserwacjom z takiego wzoru przypisać etykiety.
Jest to zadanie zwykle wykonywanie przez człowieka.
Tradycyjne algorytmy uczenia nadzorowanego do skutecznego działania potrzebują dużych zbiorów uczących. 
Jednak przygotowanie takich zbiorów jest zasobochłonne.
Rozwiązaniem tego problemu moga być algorytmu pół-nadzorowane.
Projektowane są one pod kątem optymalizacji działania w warunkach małego zbioru uczącego i dużego zbioru obserwacji bez etykiet.
Algorytmy pół-nadzorowane zakładają dwie własności danych na których pracują.
Po pierwsze zakładają spójność danych, co oznacza, że punktu leżące blisko siebie w przestrzeni problemu najprawdopodobniej mają tą samą etykietę.
Po drugie, jeśli dane formułują dającą się odgraniczyć od innych punktów strukturę, to jest wysoce prawdopodobne, że noszą taką samą etykietę\cite{Zhou2004}.
Poniżej opisane zostaną dwa z nich, które zostały wykorzystane przez autora.
\subsubsection{Sformułowanie problemu}
Dany jest zbiór oberwacji \(X = \{x_1, ..., x_l, x_{l+1}, ..., x_n\}\) oraz zbiór etykiet \(\mathcal{L} = \{1, ... ,c\}\).
Pierwsze \(l\) przypadków ma przypisaną którąś z etykiet ze zbioru \(\mathcal{L}\). Celem jest przypisanie odpowiednich oznaczeń dla pozostałych punktów \(x_u(l  < u \leqslant n)\).
Do rozwiązania tak postawionego problemu w obu algorytmach należy zacząć od skonstruowania grafu, gdzie krawędzi między dwoma wierzchołkami \(i, j\) przypisana jest waga, która zależy od wyboru odpowiedniej funkcji jądra (\emph{ang. kernel function}. W pracy wykorzystano dwie: \emph{Radial Basis Function} oraz funkcję opartą o algorytm k Najbliższych Sąsiadów. Jeśli odległość euklidesową między punktami \(x_1\) i \(x_2\) oznaczymy jako \(||x_1 - x_2||\), to wartość funkcja dla tych obserwacji wyrażona jest wzorem \(exp(\gamma(||x_1 - x_2||^2)\).
Druga funkcja równa się jedności jeśli punkt jest w zbiorze k Najbliższych sąsiadów drugiego w przeciwnym wypadku wynosi 0, co można to zapisać \(1[x_1 \in kNN(x_2)]\). 
Wybierając pierwszą funkcję należy mieć na uwadze, że powstaje całkowicie połączony graf, przez co jego macierzowa reprezentacja jest wymagająca, jeśli chodzi o miejsce w pamięci komputera.
Dlatego też, rozwiązanie to mało skalowalne, co przy dużych zbiorach obserwacji może być kłopotliwe.
Wybór drugiej funkcji powstaje do powstania macierzy składające się w przeważającej części z zer, dlatego też można wybrać reprezentację pochłaniającą znacznie mniej pamięci.
\subsubsection{Propagacja etykiet (label propagation)}
Po wyborze funckji jądra i zbudowaniu grafu algorytmie Propagacji Etykiet należy pszystąpić do konstrukcji macierzy przejścia \(T\).
\[T_{ij} = P(j \rightarrow i) = \frac{w_{ij}}{\sum^n_{k=1}w_{kj}}\]
Liczbę \(T_{ij}\) możemy traktować jako prawdopodobieństwo propagacji etykiety z wezła \(j\) na węzeł \(i\). Zdefiniujmy macierz \(Y\) o wymiarach \((n \times |\mathcal{L}|)\), gdzie wiersz odpowiada danej obserwacji, zaś liczby w poszczególnych kolumnach (po normalizacji) interpretujemy jako miarę prawdopodobieństwa przypisania obserwacji określonej etykiety.
Sam algorytm jest natury iteracyjnej i składa się z trzech kroków:
\begin{enumerate}
	\item Propagowania etykiet \(Y \leftarrow TY\).
	\item Normalizacji każdego wiersza macierzy Y.
	\item Odtworzeniu wierszy \(Y\) z pierwotnie ustalonymi etykietami.
\end{enumerate}
Po spełnieniu warunku stop przypisujemy obserwacjom etykiety na podstawie macierzy Y\cite{Zhu2002}.
\subsubsection{Przeciąganie etykiet (label spreading)}
Zdefiniujmy macierz \(F\) o wymiarach \((n \times |\mathcal{L}|)\) , która pełni podobną rolę do macierzy \(Y\) w algorytmie Propagacji Etykiet.
Natomiast w algorytmie Przeciągania etykiet macierzy Y zdefiniowana jest nieco odmiennie.
Macierz składa się w większości zer. W pierwszych l wierszach występuje 1 na pozycji reprezentującej etykietę pierwotnie przypisaną obserwacji.
Podobnie jak w poprzednim algorytmie budujemy macierz sąsiedztwa między wierzchołkami stworzonego grafu, gdzie wartości je wypełniające zależą od wybranej funkcji jądra.
Nazwijmy ją macierzą \(W\). 
Następnie tworzymy macierz \(S = D^{-1/2}WD^{-1/2}\), gdzie \(D\) to macierz diagonalna z elementami będącymi sumą \(i\)-tego wiersza w macierzy \(W\).
Następnie przystępujemy do kroku iteracyjnego obliczając kolejne macierze \(F\): \(F(t+1) = \alpha SF(t) + (1 - \alpha)Y\), gdzie \(a \in [0,1]\).
Ostatecznie przypisujemy obserwacji etykietę na podstawie wartości macierzy \(F\) tak, że \(y_i = argmax_{j\leqslant |\mathcal{L}|} F_{ij}\). 
\subsection{Indeksy}
\section{Wyniki}
\section{Dyskusja}
\bibliographystyle{plain}

\bibliography{magisterka}

\end{document}
